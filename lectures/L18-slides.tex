\input{configuration}

\title{Lecture 18 --- Testing \& Error-Checking }

\author{J. Zarnett\\
\texttt{jzarnett@uwaterloo.ca}}
\institute{Department of Electrical and Computer Engineering \\
  University of Waterloo}
\date{\today}

\begin{document}

\begin{frame}
  \titlepage
  
  \begin{center}
  \small{Acknowledgments: D.W. Harder, W.D. Bishop}
  \end{center}
 \end{frame}

\begin{frame}
\frametitle{Programming Errors}

A programming error is any kind of mistake that occurs during the development and/or maintenance of a computer program.

Even experienced developers will make mistakes. Nobody's perfect.

Errors can be classified into three broad categories:
\begin{itemize}
	\item Compile-time errors; and
	\item Run-time errors
	\item Logic errors
\end{itemize}
\end{frame}

\begin{frame}
\frametitle{Compile-Time Errors}
\textbf{Compile-Time Errors} are:
\begin{itemize}
	\item Detected when the program is compiled
	\item Programming errors that prevent compilation of the program
	\item An incorrect use of the programming language
\end{itemize}
Examples include:
		\begin{itemize}
			\item Forgetting a semi-colon
			\item Misspelling a method name
			\item Using an operand of an inappropriate type
			\item Having unmatched brackets
			\item Using a variable not in scope
		\end{itemize}


\end{frame}

\begin{frame}
\frametitle{Run-Time Errors}
\textbf{Run-Time Errors} are:
\begin{itemize}
	\item Programming errors that allow compilation but result in incorrect behaviour when the program is run
	\item Detected when the program is executing
	\item Not detectable at compile-time
\end{itemize}

Examples include:
		\begin{itemize}
			\item Division by zero
			\item An infinite loop
			\item Parsing a string input to integer when the user typed in something that's not a valid integer
		\end{itemize}

\end{frame}

\begin{frame}
\frametitle{Logic Errors}
\textbf{Logic Errors} are:
\begin{itemize}
	\item Programming errors that allow compilation and execution, but produce incorrect results
	\item Detected when the program is executing or after it is finished
	\item Not detectable at compile-time
	\item When what you wrote is valid but not what you meant to write
\end{itemize}

Examples include:
		\begin{itemize}
			\item Accessing index 1 of an array when you wanted index 0.
			\item Putting in the wrong mathematical formula, such as \texttt{a / b + 1} instead of \texttt{a / (b+1)}
			\item Using the wrong algorithm to solve a problem
		\end{itemize}

\end{frame}

\begin{frame}
\frametitle{The C\# Approach to Errors}
C\# tries to detect as many types of errors as possible at compile-time.

C\# imposes a very strict set of rules on the use of programming language constructs.


The advantage is that C\# detects more errors at compile-time.

The disadvantages are that the language forces us to be more explicit about things at compile time and run-time performance is slower.

(Other languages like C++ may allow accessing an entry that's past the end of the array unintentionally!)

\end{frame}

\begin{frame}
\frametitle{Finding All Errors}
If all errors were detectable at compile-time, could programmers develop error-free programs?

Possibly; an error that is detected in advance is likely to be fixed.\\
\quad Assumption: cost of fixing the error less than cost of error's impact.


This is a matter of engineering judgement...

Even in life and safety critical situations, like nuclear weapons, an assessment is done to determine the risks and how to deal with them.\\
\quad Such analysis must decide on what an ``acceptable'' level of risk is.

\end{frame}

\begin{frame}
\frametitle{Finding All Errors}
Can we even predict all errors in advance?

Unlikely. There are multiple reasons:\\
\quad User input is unpredictable at compile-time.\\
\quad Old or damaged hardware can result in strange behaviour.\\
\quad The behaviour of complex systems is difficult to model.\\
\quad ... and so on. This is not an exhaustive list.

If someone invented a programming language that allowed no possibility of error at run-time, it would probably not be useful.

Analogy: a computer that is unplugged and buried inside 2m of solid concrete is secure, but not useful...

\end{frame}

\begin{frame}
\frametitle{Finding Run-Time Errors}
Acknowledging that we cannot find all errors in advance, it's still worthwhile to try to find errors.

If we find errors, we can remove them, and get a little more certainty that the software does what it says it should.

One basic strategy for finding errors is: \alert{software testing}.

You surely have done some testing already for your assignments.\\
\quad Any time you provided input and looked at the output was a test.

\end{frame}

\begin{frame}
\frametitle{Software Testing}
Software testing is a complex subject that will be covered more next term in ECE~155, and then in the Software Testing course later.

Software testing, put simply, is a process in which we check if a piece of code does what it's supposed to do.

Remember that software takes some inputs and returns outputs.\\
\quad Thus, testing is:\\
\quad \quad Provide inputs, examine the output, and check if it's correct.

\end{frame}

\begin{frame}
\frametitle{Testing: Input \& Output}
Recall the \texttt{factorial()} program.

We can test it by providing 5 as an input to the function.

Then examine the output (return value, 120) to see if it's correct.

Critically, to know if the value returned is correct, we have to know what the correct value is independently of what the function returns.

When it's a mathematical problem, it's easy to know what's correct.\\
\quad Other problem domains may be much more difficult...

\end{frame}

\begin{frame}
\frametitle{Testing Your Code}
At this stage, when you write some code, you may test it by trying out a few different inputs to see what the output is.

A pairing of inputs and outputs makes a \alert{test case}.\\
\quad Example: input 5, output 120

To check the code you have written, you will execute a set of test cases to check if your code does what it should. 

\end{frame}

\begin{frame}
\frametitle{Valid and Invalid Input}
All of the preceding discussion assumes that the input is valid.

A major source of programming errors is invalid input.\\
\quad This could be intentional or unintentional on the part of the user.

To reduce the possibility of negative outcomes from errors, we can implement \alert{error-checking} in the code.

An implementation of error checking detects invalid input and deals with it in some way.

\end{frame}

\begin{frame}
\frametitle{Why Check for Errors?}

It's not strictly necessary to check for errors; we can just ignore them all and forge ahead regardless.

If we ignore an error, however, we are likely to get incorrect output. 

Maybe the program keeps running; its state is corrupted and that may manifest in incorrect output or behaviour later.

Some errors cannot be ignored; they will result in the termination of the program. Example: stack overflow in a recursive function.

Ideally, check for errors in advance, such as ensuring that the divisor is not zero before a division operation.


\end{frame}

\begin{frame}[fragile]
\frametitle{Checking for Errors In Advance}

Checking for an error in advance might allow us to deal with something before it becomes a bigger problem.

\begin{verbatim}
static void printGreetings ( int n )
{
    while ( n != 0 )
    {
        Console.WriteLine( "Greetings!" );
        n--;
    }
}
\end{verbatim}
If \texttt{n} starts as less than 0, this will be an infinite loop.\\
\quad Reporting the error is much better than the program getting stuck.


\end{frame}

\begin{frame}[fragile]
\frametitle{Checking the Precondition}
The precondition of a function, as you recall, is whatever is assumed to be true when a function is called.

Check these assumptions at the start of the function!

If the precondition is that the value of \texttt{n} must be greater than or equal to 1, we can implement a check for that in code:

\begin{verbatim}
// Precondition: the value of n must be >= 1
static int factorial ( int n )
{
     if ( n < 1 )
     {
          // Error has occurred
     }
// Rest of the implementation not shown for space reasons
}
\end{verbatim}

\end{frame}

\begin{frame}
\frametitle{Reporting the Error}
Detecting a violation of the precondition is not enough on its own; we also have to alert the user that something is wrong.

Consider that in \texttt{factorial()} if the input was outside of the valid range (integers >= 1), this is an \textit{exception} to what is expected.

The C\# concept for this is the \texttt{Exception}.

Instead of returning a value when an unexpected situation occurs, instead the code \alert{throw}s an exception.

\end{frame}

\begin{frame}
\frametitle{Throwing Exceptions}

Here are a few reasons why a programmer might throw an exception:

\begin{itemize}
	\item To provide the programmer with detailed information on the event that caused the error to occur
	\item To halt program execution at the point at which the error manifests itself
	\item To prevent further damage to data used by the program
\end{itemize}

When a program throws an exception, it indicates that it has encountered a situation that it is not immediately ready to handle.

\end{frame}

\begin{frame}[fragile]
\frametitle{Throwing an Exception}

When throwing an exception, we can provide some text to explain what went wrong or why we are throwing the exception.

\begin{verbatim}
// Precondition: the value of n must be >= 1
static int factorial ( int n )
{
     if ( n < 1 )
     {
        throw new Exception( "n must be >= 1." );
     }
// Rest of the implementation not shown for space reasons
}
\end{verbatim}

Note: it's necessary to use the \texttt{new} keyword in creating an exception.

\end{frame}

\begin{frame}
\frametitle{Exceptions}

The keyword \texttt{throw} is similar to \texttt{return}, although the path back is slightly different.

If there is no code set up to \alert{catch} an exception that has been thrown, the exception will terminate the program.

In the next lecture, we'll examine in more detail what happens when an exception is thrown.

We will also learn how to handle (catch) an exception, and what to do with one if we do catch it.


\end{frame}

\end{document}

