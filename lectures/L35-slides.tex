\input{configuration}

\title{Lecture 34 --- Abstract Classes }

\author{J. Zarnett\\
\texttt{jzarnett@uwaterloo.ca}}
\institute{Department of Electrical and Computer Engineering \\
  University of Waterloo}
\date{\today}

\begin{document}

\begin{frame}
  \titlepage
  
  \begin{center}
  \small{Acknowledgments: W.D. Bishop}
  \end{center}
\end{frame}



\begin{frame}
\frametitle{The Shape of Things to Come}

In an earlier example we considered the idea of a \texttt{Shape} class as a parent class.

It had several descendants, such as \texttt{circle, rectangle, triangle}...

And suppose that they should all have \texttt{double ComputeArea( )}.

But it makes no real sense to create an instance of a \texttt{Shape} object directly -- it needs to be something.

The solution is an \alert{abstract class}.

\end{frame}



\begin{frame}
\frametitle{Abstract Class}

An abstract class can only be used as a base to derive other classes from.

You cannot create an instance of an abstract class, because it is ``incomplete''.

It is, however, a way to create a partial class definition that can be extended as is necessary. 

To be abstract, the class must contain one \alert{pure virtual function}.

\end{frame}



\begin{frame}
\frametitle{Pure Virtual Function}

To declare a pure virtual function, we use the keyword \texttt{virtual} as well as an assignment to zero.

\texttt{virtual double ComputeArea( ) = 0;}

You may have multiple pure virtual functions in a class, and they may be of any type and take any parameters.

\end{frame}



\begin{frame}
\frametitle{Abstract Class}

A class that extends an abstract class can go down one of two routes.

\begin{enumerate}
	\item Either it implements all the pure virtual functions; or
	\item It will itself also be an abstract class.
\end{enumerate}


\end{frame}



\begin{frame}
\frametitle{Abstract Class Examples}

Example: abstract class $A$ has 5 pure virtual methods. 

$B$ extends $A$ and implements all 5: $B$ is not an abstract class.

$C$ extends $A$ and implements 4 of the 5: $C$ is abstract.

$D$ extends $C$ and does not implement the ``missing'' abstract function: $D$ is abstract.

$E$ extends $D$ and implements the last function: $E$ is not abstract.

In theory, additional pure virtual functions could be introduced in, say, $D$, which $E$ would also have to implement.

\end{frame}



\begin{frame}
\frametitle{Interfaces}

Other programming languages like Java and C\# have the concept of an \alert{interface}.

An interface is like an abstract class that has no methods or fields implemented in it at all; all functions are virtual.

Thus, the same behaviour in C++ can be achieved with an abstract class with all pure virtual functions.

\end{frame}


\end{document}

