\input{configuration}

\title{Lecture 33 --- Sorting, Continued }

\author{J. Zarnett\\
\texttt{jzarnett@uwaterloo.ca}}
\institute{Department of Electrical and Computer Engineering \\
  University of Waterloo}
\date{\today}

\begin{document}

\begin{frame}
  \titlepage
  
  \begin{center}
  \small{Acknowledgments: W.D. Bishop}
  \end{center}
\end{frame}

\begin{frame}
\frametitle{Insertion Sort}
Last time, we looked at the selection sort.\\
\quad Now we'll look at a similar sorting algorithm: insertion sort.

Recall the scenario: you have been taking notes for a course on loose-leaf paper and have been numbering the pages.

You accidentally knock over the pile of pages and they scatter and fly everywhere in your room, landing all over.

You have an unsorted pile and a (currently-empty) sorted pile.

\end{frame}


\begin{frame}
\frametitle{Insertion Sort}

Take the first page off the unsorted pile and put it in the sorted pile.\\
\quad It may be any number. Suppose it's number 22.

Grab the next page from the unsorted pile and look at its number.\\
\quad Let's say it's number 4.

Add it to the sorted pile in the appropriate order.\\
\quad In this case, add number 4 in front of number 22.

Take the next page from the unsorted pile (number 17).\\
\quad Put it into the sorted pile in its place (between 4 and 22).

Repeat this procedure until all the pages are sorted again.

\end{frame}

\begin{frame}
\frametitle{Insertion Sort Algorithm}
This algorithm using extra memory (not in-place) looks like:

For an input array of capacity $n$, allocate an output array of capacity $n$.\\
Then, for each entry of the input array:

\begin{enumerate}
	\item Remove the entry from the input array
	\item Find where to put it in the output array
	\item Put it in the output array at that position
\end{enumerate}

\end{frame}

\begin{frame}
\frametitle{Insertion Sort In-Place}

Insertion Sort can also be done in-place:

Divide the unsorted array (conceptually) into two smaller arrays:\\
\quad The 1$^{st}$ contains entries processed and inserted in order.\\
\quad The 2$^{nd}$ contains entries that have not yet been processed.

We iterate over all entries of the array, taking entries from the second of the small arrays, processing them, and adding them to the first.

We're finished when the 1$^{st}$ array has grown to the size of the input.\\
\quad This means all elements are processed and sorted.

\end{frame}


\begin{frame}
\frametitle{Insertion Sort Algorithm}
Given an array of $n$ entries:

For $index$ from 1 to $(n-1)$:
\begin{enumerate}
	\item Set $temp$ to the value of the array at $index$
	\item For $pos$ from $index$ to 1
		\begin{enumerate}[a)]
			\item Check if $temp$ is greater than the value of the array at $(pos-1)$\\
			\quad\quad\quad If yes, the exit the inner loop\\
			\quad\quad\quad If no, swap value of array at $pos$ with the value at $(pos-1)$
		\end{enumerate}
	\item Set value of array at $pos$ to $temp$
\end{enumerate}

\end{frame}


\begin{frame}
\frametitle{Insertion Sort Illustration}

Consider sorting \{12, 36, 20, -1\} using the insertion sort.

After iteration $x$, the first $x+1$ values have been sorted.\\
\quad Progress is shown by the red highlighting.

Iteration 1: Array Contents = \{12, 36, 20, -1\}\\
\quad Compare 36 to 12\\
\quad Exit inner loop due to larger value\\
\quad Store 36 at index 1

Iteration 2: Array Contents = \{\alert{12}, \alert{36}, 20, - 1\}\\
\quad Compare 20 to 36\\
\quad Store 36 at index 2\\
\quad Compare 20 to 12\\
\quad Exit inner loop due to larger value\\
\quad Store 20 at index 1


\end{frame}

\begin{frame}
\frametitle{Insertion Sort Illustration}

Iteration 3: Array Contents = \{\alert{12}, \alert{20}, \alert{36}, -1\}\\
\quad Compare -1 to 36 \\
\quad Store 36 at index 3\\
\quad Compare -1 to 20\\
\quad Store 20 at index 2\\
\quad Compare -1 to 12\\
\quad Store 12 at index 1\\
\quad Exit inner loop due to start of array\\
\quad Store -1 at index 0

The array, after sorting, contains \{-1, 12, 20, 36\}.

\end{frame}

\begin{frame}
\frametitle{Larger Array Example}

Consider sorting \{12, 36, 42, -1, 99, 20, 10, 19, 70\}.

	\{12, 36, 42, -1, 99, 20, 10, 19, 70\} 	// Initial array values\\
	\{\alert{12, 36}, 42, -1, 99, 20, 10, 19, 70\}	// After iteration 1\\
	\{\alert{12, 36, 42}, -1, 99, 20, 10, 19, 70\}	// After iteration 2\\
	\{\alert{-1, 12, 36, 42}, 99, 20, 10, 19, 70\}	// After iteration 3\\
	\{\alert{-1, 12, 36, 42, 99}, 20, 10, 19, 70\}	// After iteration 4\\
	\{\alert{-1, 12, 20, 36, 42, 99}, 10, 19, 70\}	// After iteration 5\\
	\{\alert{-1, 10, 12, 20, 36, 42, 99}, 19, 70\}	// After iteration 6\\
	\{\alert{-1, 10, 12, 19, 20, 36, 42, 99}, 70\}	// After iteration 7\\
 	\{\alert{-1, 10, 12, 19, 20, 36, 42, 70, 99}\}	// After iteration 8

Note that a total of 8 iterations are required to sort 9 values.

In general, insertion sort requires $n - 1$ iterations to sort $n$ values.

\end{frame}


\begin{frame}[fragile]
\frametitle{Insertion Sort Code}

{\scriptsize
\begin{verbatim}
public static void InsertionSort( int[] data )
{
    for( int index = 1; index < data.Length; index++ )
    {
        int pos;
        int temp = data[index];

        for( pos = index; pos > 0; pos-- )
        {
            if( temp > data[pos - 1] )
            {
                break;
            }
            else
            {
                data[pos] = data[(pos - 1)];
            }
        }
        data[pos] = temp;
    }
}
\end{verbatim}
}
\end{frame}


\begin{frame}
\frametitle{Sorting Linked Lists}

The selection sort and the insertion sort work well on arrays, but how can they be used on linked lists?

To use these sorting algorithms, you simply need to keep track of an unsorted list of elements and a sorted list of elements.

Both can be used on linked lists without needing additional storage.\\
\quad An element will only exist in one list at a time.\\
\quad Therefore, the amount of storage required is constant.

\end{frame}


\begin{frame}
\frametitle{Selection Sort for a Linked List}

Given a list of $n$ elements:

For each of the $n$ elements:
\begin{enumerate}
	\item Remove the minimum element from the unsorted list
	\item Append that element to the sorted list.
\end{enumerate}

The difficulty: removing the minimum element from the unsorted list.

\end{frame}

\begin{frame}[fragile]
\frametitle{Linked List Selection Sort}
{\scriptsize
\begin{verbatim}
public void SelectionSort( )
{
    List sortedList = new List( );

    Element tmp = DeleteMin( );
    while( tmp != null )	
    {
        sortedList.Append( tmp );
        tmp = DeleteMin( );
    }

    head = sortedList.head;
    tail = sortedList.tail;
}
\end{verbatim}
}

The \texttt{SelectionSort} method uses a private implementation of an \texttt{Append} method that has access to the \texttt{Element} class.  

A public version of the \texttt{Append} method might also exist that uses an integer to construct a new \texttt{Element} and append it to a list.

\end{frame}


\begin{frame}[fragile]
\frametitle{The \texttt{DeleteMin} Method}
The \texttt{DeleteMin} method is too big to put on the slides.

There are a number of special case handling blocks, such as when the list is empty or contains only one element...

[In class demo: the code for this method]

\end{frame}

\begin{frame}[fragile]
\frametitle{Testing Selection Sort}

\begin{verbatim}
List myList = new List( );
myList.Append( 12 );
myList.Append( 36 );
myList.Append( 42 );
myList.Append( -1 );
myList.Append( 99 );
myList.Append( 20 );
myList.Append( 10 );
myList.Append( 19 );
myList.Append( 70 );

Console.WriteLine( "\nArray before sorting" );
Console.WriteLine( myList );
myList.SelectionSort( );
Console.WriteLine( "Array after sorting" );
Console.WriteLine( myList );
\end{verbatim}

\end{frame}


\begin{frame}
\frametitle{Insertion Sort for a Linked List}

Given a list of $n$ elements:

For each of the $n$ elements:
\begin{enumerate}
	\item Remove the head element from the unsorted list
	\item Insert that element at the correct position in the sorted list.
\end{enumerate}

The difficulty: adding the element to the sorted list.

\end{frame}

\begin{frame}[fragile]
\frametitle{Linked List Insertion Sort}

\begin{verbatim}
public void InsertionSort( )
{
    List sortedList = new List( );

    Element tmp = DeleteHead( );
    while( tmp != null )
    {
        sortedList.InsertInOrder( tmp );
        tmp = DeleteHead( );
    }

    head = sortedList.head;
    tail = sortedList.tail;
}
\end{verbatim}

\end{frame}


\begin{frame}[fragile]
\frametitle{The \texttt{InsertInOrder} Method}
The \texttt{InsertInOrder} method is also too big to put on the slides.

Like \texttt{DeleteMin}, there are a number of special case handling blocks, such as when the list is empty or contains only one element...

[In class demo: the code for this method]

\end{frame}

\begin{frame}[fragile]
\frametitle{Testing Selection Sort}

\begin{verbatim}
List myList = new List( );
myList.Append( 12 );
myList.Append( 36 );
myList.Append( 42 );
myList.Append( -1 );
myList.Append( 99 );
myList.Append( 20 );
myList.Append( 10 );
myList.Append( 19 );
myList.Append( 70 );

Console.WriteLine( "\nArray before sorting" );
Console.WriteLine( myList );
myList.InsertionSort( );
Console.WriteLine( "Array after sorting" );
Console.WriteLine( myList );
\end{verbatim}

\end{frame}


\begin{frame}
\frametitle{Planning Ahead}

Constructing a sorted linked list is often more efficient than sorting a linked list after creation.

Insertion takes more time, but then sorting is not required.

This minimizes the number of traversals of the linked list.

Given the existence of an \texttt{InsertInOrder} method, simply use that every time an element is added.

As a general rule, you should always build sorted data structures if you plan to search them.


\end{frame}


\begin{frame}
\frametitle{Sorting Through the Ages}
If you think sorting a linked list is tricky, imagine the difficulty of sorting data stored sequentially on magnetic tape units.

An interesting article on sorting data on magnetic tape units:
\url{http://portal.acm.org/ft_gateway.cfm?id=808961&type=pdf}
	
For an estimated 20 000 records of 200 characters, the sorting algorithm described required 270 minutes to complete.

That's 4.5 hours to sort 4 MB of character records.
\end{frame}

\begin{frame}
\frametitle{Sorting Performance Comparison}

The performance of selection sort, insertion sort, and sorted lists can be easily compared.

If a list contains 100 000 random integers:
\begin{center}
\begin{tabular}{l|l}
\textbf{Operation} & \textbf{Time (s)}\\ \hline
Linked list creation time & 0:00.01\\ \hline
Selection sort time		& 1:45.07\\ \hline
Insertion sort time		& 1:09.26\\ \hline
Sorted list creation time	& 1:10.12\\
\end{tabular}
\end{center}

\end{frame}

\begin{frame}
\frametitle{Sorting in Reality}
Outside of an academic context, it is rare that you will implement your own search algorithm, but doing so is important for learning.

In practice, when you use the built-in collection types of the language, they have \texttt{Sort} methods that will sort the data for you.

For integers and other built-in types, you may use this \texttt{Sort} method and the system will be able to sort the data correctly.

But what about programmer-defined types like classes?

\end{frame}


\begin{frame}
\frametitle{Sorting Programmer-Defined Types}
To sort programmer-defined types, we need to decide on a sort order.\\
\quad We might sort students by last name, or student ID number... 

And we need to find a way to tell the system what we chose.\\
The system can't know in advance what types we are going to define...

Next time, we'll examine how to do that.
\end{frame}




\end{document}

