\input{configuration}

\title{Lecture 32 --- Library Functions }

\author{J. Zarnett\\
\texttt{jzarnett@uwaterloo.ca}}
\institute{Department of Electrical and Computer Engineering \\
  University of Waterloo}
\date{\today}

\begin{document}

\begin{frame}
  \titlepage
  
  \begin{center}
  \small{Acknowledgments: W.D. Bishop}
  \end{center}
\end{frame}


\begin{frame}
\frametitle{Library Functions}
The system provides a number of classes and types (in various namespaces, like \texttt{std}) for programmers' use.

We often ask students, as exercises, to implement functions like absolute value or exponentiation.

In practical situations, these functions are already available to us as static methods of specific classes.

Many of the built-in classes are in the \texttt{std} namespace.

\end{frame}

\begin{frame}
\frametitle{\texttt{Math} Class}
A number of different mathematical operations:

\begin{center}
\includegraphics[width=0.65\textwidth]{images/math.png}
\end{center}

Credit: \textit{Absolute C++} by Walter Savitch.


\end{frame}

\begin{frame}[fragile]
\frametitle{Example: Using the \texttt{cmath} Functions}

To make use of a function like \texttt{sqrt}, we need to include the appropriate library header.

\texttt{\#include<iostream>} has appeared frequently in our code.

This includes the functions of the \texttt{iostream} header.

Now, to use the square root function, use \texttt{\#include<cmath>}.

Then we can simply use this function:\\
\quad \texttt{double root = sqrt( number );}

\end{frame}

\begin{frame}
\frametitle{Library Methods}

The examples shown in the previous slides show library (system-provided) functions.

Although we may ask you to work with these methods or to implement them, it is rarely a good practice to ``reinvent the wheel''.

If you attempt to reinvent something that is already created, there is a good chance you will make some mistakes, even if they are obscure.\\
\quad For an interesting real-world example, see:\\
\quad \url{http://blog.codinghorror.com/whats-wrong-with-turkey/}

When a library function exists to do the operation you want to perform, make use of that library function.

Be sure, however, to check the documentation to learn how to call the function properly (e.g., order of parameters).

\end{frame}


\end{document}

