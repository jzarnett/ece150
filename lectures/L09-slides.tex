\input{configuration}

\title{Lecture 9 --- Loops II}

\author{J. Zarnett\\
\texttt{jzarnett@uwaterloo.ca}}
\institute{Department of Electrical and Computer Engineering \\
  University of Waterloo}
\date{\today}

\begin{document}

\begin{frame}
  \titlepage
  
  \begin{center}
  \small{Acknowledgments: D.W. Harder, W.D. Bishop}
  \end{center}
 \end{frame}
 

\begin{frame}
\frametitle{Loops, Continued}
Last time, we examined the \texttt{while} and \texttt{do-while} loops.

We also mentioned, but did not examine, the third kind of loop:\\
\quad the \texttt{for} loop.

Let's start by re-examining a \texttt{while} loop.

\end{frame}


\begin{frame}[fragile]
\frametitle{Countdown Loop}

\begin{verbatim}
int countdown = 10;
while ( countdown > 0 ) 
{
    Console.WriteLine( countdown );
    countdown--;
}
\end{verbatim}

This is a counted loop. Observe three statements:
\begin{enumerate}
	\item \texttt{int countdown = 10;}\\
		\quad Declares and initializes the counter variable.
	\item \texttt{while (countdown > 0)}\\
		\quad The loop test condition.
	\item \texttt{countdown{-}{-};}\\
		\quad Decrement the counter variable.
\end{enumerate}

\end{frame}

\begin{frame}
\frametitle{Counted Loops}
Counted loops are very common. The \texttt{for} loop syntax puts all three of those statements in one line. The proper names for these elements:

\begin{enumerate}
	\item \texttt{int countdown = 10;}\\
		\quad \alert{Initialization}
	\item \texttt{while (countdown > 0)}\\
		\quad \alert{Condition}
	\item \texttt{countdown{-}{-};}\\
		\quad \alert{Modification}
\end{enumerate}

Now we can see the syntax of the for loop.

\end{frame}

\begin{frame}
\frametitle{The \texttt{for} Loop}

for ( \textit{initialization}; \textit{condition}; \textit{modification} )

Using the countdown example:\\
\quad \texttt{for ( int countdown = 10; countdown > 0; countdown{-}{-} )}

And as always, the declaration is followed by the \{ and \} braces to enclose the loop body (statement block).

\end{frame}

\begin{frame}[fragile]
\frametitle{Rewriting the Countdown Loop}

Let's rewrite the countdown loop using \texttt{for}:

\begin{verbatim}
for ( int countdown = 10; countdown > 0; countdown-- )
{
    Console.WriteLine( countdown );
}
\end{verbatim}

Important note: there is no semicolon after the modification expression. A \texttt{for} loop has exactly two semicolons.

(Technical explanation: the three elements are expressions and not statements, and that's why no semicolon is required.)

\end{frame}

\begin{frame}[fragile]
\frametitle{Execution Order of the \texttt{for} Loop}

Let's examine the execution order of this loop:

\begin{verbatim}
for ( int i = 0; i < 1; ++i )
{
    // Loop Body
}
\end{verbatim}

\begin{enumerate}
	\item The initialization: \texttt{int i = 0;}
	\item The condition is checked: \texttt{i < 1;}
	\item The condition is true, so the loop body executes
	\item The modification: \texttt{++i}
	\item Control goes back to the start of the loop; condition is checked
	\item The condition is false, so the loop body is skipped
\end{enumerate}


\end{frame}

\begin{frame}
\frametitle{For Loop Execution Comments}

Like the \texttt{while} loop, the \texttt{for} loop is a pretest loop.

If the condition is not satisfied, the loop body does not execute.\\
\quad This may mean the loop executes zero times.

\end{frame}

\begin{frame}[fragile]
\frametitle{Optional Parts of the \texttt{for} Loop}

Technically, each of the 3 elements of the \texttt{for} statement are optional.

In this example, initialization is missing:

\begin{verbatim}
int countdown = int.Parse(Console.ReadLine());
for ( ; countdown > 0; countdown-- )
{
    Console.WriteLine( countdown );
}
\end{verbatim}

The semicolon is still needed after the spot where the initialization would otherwise go.

\end{frame}

\begin{frame}[fragile]
\frametitle{Optional Parts of the \texttt{for} Loop}

In this example, the modification expression is elsewhere in the loop:

\begin{verbatim}
for ( int countdown = 10; countdown > 0; )
{
    Console.WriteLine( countdown );
    countdown--;
}
\end{verbatim}

\end{frame}

\begin{frame}[fragile]
\frametitle{Optional Parts of the \texttt{for} Loop}

Let's combine the last two examples:

\begin{verbatim}
int countdown = int.Parse(Console.ReadLine());
for ( ; countdown > 0; )
{
    Console.WriteLine( countdown );
    countdown--;
}
\end{verbatim}

This is also acceptable. If you use only the middle expression (condition) of the \texttt{for} loop, it works just like the \texttt{while loop}.

(In fact, any \texttt{while} loop can be trivially converted to a \texttt{for} loop by using this syntax.)

\end{frame}

\begin{frame}[fragile]
\frametitle{Optional Parts of the \texttt{for} Loop}

And if we delete the condition, too?

\begin{verbatim}
int countdown = int.Parse( Console.ReadLine() );
for ( ; ; )
{
    Console.WriteLine( countdown );
    countdown--;
    if ( countdown < 0 )
    {
        break;
    }    
}
\end{verbatim}

If no condition is specified, it's an infinite loop; the same as if we wrote \texttt{while (true)}.

\end{frame}

\begin{frame}[fragile]
\frametitle{For Loop Flexibility}

The elements of the \texttt{for} loop are pretty flexible. The initialization and modification expressions may do more than one thing.

The expressions are separated by commas.

\begin{verbatim}
for ( int x = 10, int y = 0; x > 0; x--, y++ )
{
    // Loop Body
}
\end{verbatim}

This is uncommon and may be confusing, especially to non-experts. It is not recommended as a good practice.

\end{frame}

\begin{frame}[fragile]
\frametitle{Pitfall: Extra Semicolon}

Potential pitfall: there is an extra semicolon after the closing bracket.

\begin{verbatim}
for ( int x = 10; x > 0; x-- );
    Console.WriteLine( "Hello World" );
\end{verbatim}

The for loop will run, but the \texttt{Console.WriteLine} statement will execute only once.

The semicolon creates an ``empty statement'' and that forms the body of the loop. That empty statement executes with each iteration.

Solution: like with \texttt{if}, use the \{ and \} braces to prevent this problem.

\end{frame}

\begin{frame}
\frametitle{Use of \texttt{break} and \texttt{continue}}

The \texttt{break} and \texttt{continue} statements can also appear in the body of the \texttt{for} loop as they do in the \texttt{while} loop.

As before, \texttt{break} exits from the loop and takes us to the next statement after the loop.

\texttt{continue} still means ``go back to the start of the loop'', but the modification statement is executed before the condition is checked.

\end{frame}

\begin{frame}[fragile]
\frametitle{Use of \texttt{continue} in a \texttt{for} Loop}
In this example, numbers that are multiples of 8 are not printed.
\begin{verbatim}
int numberPrinted = 0;
for ( int i = 0; i < 100; ++i )
{
    if ( i % 8 == 0 )
    {
        continue;
    }
    Console.WriteLine( i );
    numberPrinted += 1;
}
Console.Write( numberPrinted );
Console.WriteLine ( " numbers printed in total." );
\end{verbatim}

If \texttt{i} is a multiple of 8, the next iteration of the loop is started.

\end{frame}

\begin{frame}[fragile]
\frametitle{Confusing \texttt{continue} and \texttt{break}}
What if instead of using \texttt{continue} we used \texttt{break} instead?
\begin{verbatim}
int numberPrinted = 0;
for ( int i = 0; i < 100; ++i )
{
    if ( i % 8 == 0 )
    {
        break;
    }
    Console.WriteLine( i );
    numberPrinted += 1;
}
Console.Write( numberPrinted );
Console.WriteLine ( " numbers printed in total." );
\end{verbatim}



\end{frame}

\begin{frame}
\frametitle{Nested Loops}

It is legal to nest one loop statement inside another.\\
\quad Even different types (e.g., a \texttt{for} inside a \texttt{while}).

When nesting \texttt{for} loops, they should use different counter variables.\\
\quad Common choices are \texttt{i}, \texttt{j}, and \texttt{k}.\\
\quad Choosing more descriptive names can be helpful.

A \texttt{break} or \texttt{continue} statement applies to the innermost loop statement containing that statement.

(We can also put a \texttt{switch} statement inside a loop; a \texttt{break} statement at the end of an option applies to the \texttt{switch}, not the loop.)

\end{frame}

\begin{frame}[fragile]
\frametitle{Nested Loop Example}

\begin{verbatim}
for ( int i = 0; i < 10; i++ )
{
    for (int j = 0; j < 10; j++ )
    {
        Console.Write( "i = " );
        Console.Write( i );
        Console.Write( ", j = " );
        Console.Write( j );
        Console.Write( "\n" );
    }
}
\end{verbatim}

\end{frame}

\begin{frame}[fragile]
\frametitle{Nested Loop with \texttt{break} Example}

\begin{verbatim}
for ( int i = 0; i < 10; i++ ) // Outer loop
{
    for (int j = 0; j < 10; j++ ) // Inner loop
    {
        if ( j == 9 )
        {
            break; // End inner loop
        }
        Console.Write( "i = " );
        Console.Write( i );
        Console.Write( ", j = " );
        Console.Write( j );
        Console.Write( "\n" );
    }
}
\end{verbatim}

\end{frame}


\begin{frame}
\frametitle{Comments on Iteration Statements}

As you've seen, it doesn't matter what iteration statement you choose;\\
\quad It's possible to rewrite any loop as another kind.

Some tips about when to use each of the loops:

\begin{itemize}
	\item The \texttt{for} statement is a slightly better choice for deterministic loops (i.e., loops that require a fixed number of iterations)
	\item The \texttt{while} statement is slightly better for non-deterministic loops (i.e., loops that require a flexible number of iterations)
	\item The \texttt{do-while} statement is slightly better for statement blocks that must execute at least once (but is still not recommended)
\end{itemize}
\end{frame}




\end{document}

