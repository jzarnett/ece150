\input{configuration}

\title{Lecture 12 --- More About Arrays }

\author{J. Zarnett\\
\texttt{jzarnett@uwaterloo.ca}}
\institute{Department of Electrical and Computer Engineering \\
  University of Waterloo}
\date{\today}

\begin{document}

\begin{frame}
  \titlepage
  
  \begin{center}
  \small{Acknowledgments: W.D. Bishop}
  \end{center}
 \end{frame}
 
\part{The String Revisited}
\begin{frame}
\partpage
\end{frame} 
 
\begin{frame}
\frametitle{The \texttt{string} Revisited}

We have already learned about a string type, but we haven't examined it in much detail.

The string was just a bunch of text, but there is much more to it than we might think at first glance...

\texttt{string svar1; // Creates uninitialized string}\\
\texttt{string svar2 = "Literal"; // Initialized.}\\
\texttt{string svar3 = ""; // Initialized to the empty string.}

The \texttt{string} is a complex type (there's a reason it wasn't introduced in the simple types alongside \texttt{int} and \texttt{double}).

\end{frame}

\begin{frame}
\frametitle{String Concatenation}

Strings can appear in expressions using the \texttt{+} operator. It does not add up the values, but instead performs \alert{concatenation}.

\texttt{string verb = "fore" + "see";}

This means the variable \texttt{verb} contains \texttt{"foresee"}.

The use of \texttt{+=} can also be used for concatenation:\\
\quad \texttt{verb += "n";} $\rightarrow$ \texttt{verb} is now \texttt{foreseen}.

Remember that for concatenation, like an arithmetic expression, it is necessary to assign the value somewhere.

\end{frame}

\begin{frame}
\frametitle{Strings are Complex}

It turns out that the string has a member variable \texttt{Length} that tells you the number of characters in the \texttt{string}.

This information, combined with the fact that a \texttt{string} is a bunch of text characters should lead you to the conclusion that...

The \texttt{string} is really an \alert{array of \texttt{char}s}.

\end{frame}

\begin{frame}
\frametitle{An Array of \texttt{char}}

This means we can access individual characters within a \texttt{string} using their index values (just as if they were entries of an array).

If the string is \texttt{string ex1 = "example"}, the \texttt{char} at \texttt{ex1[3]} is \texttt{'m'}.

We could use a \texttt{for} loop to iterate over all the characters of the \texttt{string} if we are looking for something specific.

\end{frame}

\begin{frame}[fragile]
\frametitle{Iterating Over the \texttt{string}}

\begin{verbatim}
string s = "Hello World!";

for ( int i = 0; i < s.Length; i++ )
{
    cout << s[i] << endl;
}

for ( int j = s.Length -1; j >= 0; j-- )
{
    cout << s[j] << endl;
}
\end{verbatim}

\end{frame}

\begin{frame}
\frametitle{Mutating \texttt{string} Elements}

Characters within a \texttt{string} can be changed using index values.

This is no different from an array of \texttt{int} where we could assign \texttt{array[7] = -98;}

The string type is \textit{not} \alert{immutable}. \texttt{string} variables can be changed.


In other languages, the string is immutable; every time a string must be ``changed'', a new \texttt{string} must be created.

\end{frame}


\part{Multi-Dimensional Arrays}
\begin{frame}\partpage\end{frame}

\begin{frame}
\frametitle{Arrays of Arrays}

You may have wondered if we can have an array of any type, can we have an array of arrays? Yes!

A \alert{multi-dimensional array} is an array of array types.

The following statement declares and instantiates a multi-dimensional array of \texttt{int} named \texttt{days}: \texttt{int[12][31] days;}

Is this syntax confusing? Perhaps imagine it like this: \texttt{(int[])[]}

The type is \texttt{int[]} (in brackets) and when we declare an array of a type, write \texttt{[]} after the type.

Hence, we declare an array of type \texttt{int[]} (integer array).

\end{frame}

\begin{frame}
\frametitle{Arrays of Arrays}

\texttt{day[0]} is a reference to the first integer array\\
\texttt{day[1]} is a reference to the second integer array\\
\texttt{day[n-1]} is a reference to the nth integer array


The length of \texttt{day[0]} and \texttt{day[1]} may be different.

\end{frame}

\begin{frame}[fragile]
\frametitle{Setting up A Multi-Dimensional Array}

\begin{verbatim}
int[] daysInMonth = { 31, 28, 31, 30, 31, 30, 
                     31, 31, 30, 31, 30, 31 };

int[12][31] year;

for ( int month = 0; month < 12; ++month )
{
    year[month] = new int[ daysInMonth[month] ];
	
    for ( int day = 0; d < daysInMonth[month]; ++d )
    {
        year[month][day] = 1;
    }

}
\end{verbatim}

\end{frame}

\begin{frame}[fragile]
\frametitle{Setting up A Multi-Dimensional Array}

Now let's print out this calendar.

\begin{verbatim}
for ( int month = 0; month < 12; ++month )
{
    for ( int day = 0; d < daysInMonth[month]; ++d )
    {
        cout << year[month][day] << " ";
    }
    cout << endl;
}
\end{verbatim}

\end{frame}

\begin{frame}[fragile]
\frametitle{Further Initialization of Multi-Dimensional Arrays}
It is possible to initialize a multi-dimensional array when it is declared.

For example, the following code defines a multi-dimensional array of characters named \texttt{myChars}:

\begin{verbatim}
char[][] myChars = {
          { 'B', 'i', 'l', 'l' },
          { 'D', 'a', 'v', 'e' },
          { 'G', 'e', 'o', 'r', 'g', 'e' }
          };
\end{verbatim}

\end{frame}

\begin{frame}
\frametitle{Arrays of Arrays of Arrays...}

The multi-dimensional array examples we have shown so far are all ``two dimensional''.

We could have more, such as \texttt{int[][][] coordinates;} to describe x, y, and z co-ordinates.

Multidimensional arrays are also called ``\alert{jagged} arrays''.

\end{frame}



\begin{frame}
\frametitle{Storage in Memory}

Remember that memory is linear, so a two dimensional array has to be stored in a linear fashion.

Therefore, a two dimensional array is just an abstraction. 

Thus, the following are equivalent:\\
\quad \texttt{int[3][3];}\\
\quad \texttt{int[9];}



In C++ (and C) arrays are stored in \alert{row major} order.

Thus, all elements in the first row appear first, then all elements in the second row, etc.




\end{frame}

\end{document}

