\input{configuration}

\title{Lecture 30 --- The Linked List}

\author{J. Zarnett\\
\texttt{jzarnett@uwaterloo.ca}}
\institute{Department of Electrical and Computer Engineering \\
  University of Waterloo}
\date{\today}

\begin{document}

\begin{frame}
  \titlepage
  
  \begin{center}
  \small{Acknowledgments: W.D. Bishop}
  \end{center}
\end{frame}

\begin{frame}
\frametitle{Arrays and Collections}
We've established that we would like to have a dynamic collection that does not rely on an array.

When using an array, we have a reference to the array.\\
\quad The reference tells us where the start of the array is in memory.\\
\quad The index tells us which entry of the array to access.

In a collection, the array serves the purpose of keeping the elements next to one another so that we can move from one to the next.\\
\quad To find the next object in an array, we increment the index by 1.

\end{frame}

\begin{frame}
\frametitle{Linking Collection Elements}
If the elements are not stored in an array, given that we know where the first object is, how might we get to the next object?

Idea: use a reference to link the objects together.

In the simplest case, each object has a reference to the next object.\\
\quad A null reference indicates there is no next object.

Analogy: pirate treasure map.\\
\quad Follow the map's directions; it leads to treasure and another map.\\
\quad The 2$^{nd}$ map says how to find another treasure and a third map.\\
\quad ... and repeat until we get to the final treasure (that has no map).

\end{frame}

\begin{frame}
\frametitle{The Linked List}

Linked lists are similar in function to arrays but are more flexible:
\begin{itemize}
    \item The size of a linked list is not fixed
    \item Objects may be easily inserted or deleted from a linked list
    \item Objects may be sorted (if desired)
\end{itemize}

Linked lists are formed using self-referential classes:\\
\quad A class that has a reference to an object of the same type.

For example, a class \texttt{ListEntry} can contain one or more data fields as well as a reference to another object of type \texttt{ListEntry}.

\end{frame}


\begin{frame}[fragile]
\frametitle{Self-Referential Classes}

Wait a minute, we previously said this is a compile-time error:

\begin{verbatim}
struct Invoice
{
    public int number;
    public Date date;
    public bool paid;
    public Invoice previousInvoice;
};
\end{verbatim}

The difference is \texttt{struct} is a value type and \texttt{class} is a reference type.

A reference variable may be null; a structure variable may not.

\end{frame}

\begin{frame}[fragile]
\frametitle{Linking Objects}

Let's write a very simple \texttt{ListEntry} class.

\begin{verbatim}
public class ListEntry
{
    public int data;
    public ListEntry next;
}
\end{verbatim}

This works because \texttt{next} can be null. 

The compiler can figure out how much memory \texttt{ListEntry} needs because it knows the size of an \texttt{int} and the size of a reference (\texttt{next}).

\end{frame}

\begin{frame}
\frametitle{Creating a List of Integers}

Using the class previously defined, it is possible to create a collection of records known as a singly linked list.

For example, it is possible to define a list containing a set of integers \{3,1,6\} as follows:

\begin{center}
    \includegraphics[width=\textwidth]{images/list316.png}
\end{center}

When a reference is null, it is sometimes drawn as if it's an electrical connection to ground.

\end{frame}

\begin{frame}
\frametitle{Accessing the Linked List}
The linked list on the previous slide has no references pointing to it...

Developers often keep a reference to the head and the tail of the list:

\begin{center}
    \includegraphics[width=\textwidth]{images/listheadtail.png}
\end{center}

A tail reference is not strictly needed; it's a perfectly functional linked list with only a head reference. A tail reference can be convenient.

\end{frame}

\begin{frame}[fragile]
\frametitle{Less Minimal \texttt{ListEntry}}
{\scriptsize
\begin{verbatim}
class ListEntry
{
    int data;
    ListEntry next;
    public ListEntry( int d )
    {
        data = d;
        next = null;
    }
    public int Data
    {
        get{ return data; }
        set{ data = value; }
    }
    public ListEntry Next
    {
        get{ return next; }
        set{ next = value; }
    }
    public override string ToString( )
    {
        return( data.ToString( ) );
    }
}
\end{verbatim}
}
\end{frame}


\begin{frame}
\frametitle{The \texttt{List} Class}

Next, a class to store a reference to the head and tail of the linked list.

For the purpose of this example, this class is named \texttt{List}.

This class often provides the following features:
\begin{itemize}
    \item Member fields to refer to the head and the tail of the list
    \item A constructor
    \item A \texttt{ToString( )} method
    \item A \texttt{Clear( )} method
    \item An \texttt{Append( )} method
    \item A \texttt{DeleteFirst( )} method
    \item A \texttt{Prepend( )} method
    \item Properties for accessing the private fields
\end{itemize}

\end{frame}

\begin{frame}[fragile]
\frametitle{The \texttt{List} Class}
The \texttt{List} class is too large to fit on a single slide, but let's show the different parts.

{\scriptsize
\begin{verbatim}
class List
{
  private class ListEntry
  {
      // It is legal to have ListEntry nested inside List
      // In fact, if ListEntry makes no sense outside of
      // a List, this is good practice.
      // Not shown for space reasons
  }
  
  // Private member variables
    private ListEntry head;
    private ListEntry tail;

    // Constructor
    public List( )
    {
        head = null;
        tail = null;
    }

\end{verbatim}
}
\end{frame}

\begin{frame}[fragile]
\frametitle{The \texttt{List} Class : Properties}

{\scriptsize
\begin{verbatim}
    // Properties
    public int Head
    {
        get{ return head.Data; }
    }

    public int Tail
    {
        get{ return tail.Data; }
    }

    public bool IsEmpty
    {
        get{ return( head == null ); } 
    }
\end{verbatim}
}
\end{frame}

\begin{frame}[fragile]
\frametitle{The \texttt{List} Class : ToString}

{\scriptsize
\begin{verbatim}
    // Traverse the list starting at the head
    // Stop traversing when a null reference is found
    public override string ToString( )
    {
        string tmp = "Head -> ";

        ListEntry current = head;
        if( current == null )
        {
            tmp = "null";
        }
        while( current != null )
        {
            tmp += current + " -> ";
            current = current.Next;
        }
        return( tmp );
    }
\end{verbatim}
}

\end{frame}

\begin{frame}[fragile]
\frametitle{The \texttt{List} Class : Clear}
{\scriptsize
\begin{verbatim}
    // The implementation here is trivial.
    // By setting head and tail to null, any list entries
    // will be garbage and subject to garbage collection.
    public void Clear( )
    {
        head = null;
        tail = null;
    }
\end{verbatim}
}
\end{frame}

\begin{frame}[fragile]
\frametitle{The \texttt{List} Class : Append}

{\scriptsize
\begin{verbatim}
    // Add entry to the end of the list
    // Special case: list is empty
    public void Append( int i )
    {
        ListEntry tmp = new ListEntry( i );
        
        // Done by default, but here for clarity
        tmp.Next = null;

        if( head == null )
        {
            head = tmp;
        }
        else
        {
            tail.Next = tmp;
        }
        tail = tmp;
    }
\end{verbatim}
}

\end{frame}

\begin{frame}[fragile]
\frametitle{The \texttt{List} Class : Prepend}

{\scriptsize
\begin{verbatim}
    // Add entry to the start of the list
    // Special case: list is empty
    public void Prepend( int i )
    {
        ListEntry tmp = new ListEntry( i );

        tmp.Next = head;

        if( head == null )
        {
            tail = tmp;
        }
        head = tmp;
    }
\end{verbatim}
}
\end{frame}

\begin{frame}[fragile]
\frametitle{The \texttt{List} Class : Delete First}

{\tiny
\begin{verbatim}
   // Delete the first entry where the data is equal to i
    public void DeleteFirst( int i )
    {
        ListEntry current = head;
        ListEntry previous = null;

        while( current != null && current.Data != i )
        {
            previous = current;
            current = current.Next;
        }
        if( current == null )
        {
            throw new ArgumentException( "List entry not found" );
        }
        if( current == head ) // Special case: deleting first entry
        {
            head = current.Next;
        }
        else
        {
            previous.Next = current.Next;
        }
        if( current == tail ) // Special case: deleting last entry
        {
            tail = previous;
        }
    }
} // End of List class
\end{verbatim}
}

\end{frame}

\begin{frame}
\frametitle{List Creation Demonstration}

Let's work with this code and represent visually what the following series of statements would look like when executed.

\texttt{
\begin{enumerate}
	\item List list = new List();
	\item list.Append( 3 );
	\item list.Append( 1 );
	\item list.Append( 6 );
	\item list.Prepend( 4 );
	\item list.Prepend( 5 );
	\item list.DeleteFirst( 4 );
	\item list.Prepend( 2 );
	\item list.Clear( );
\end{enumerate}
}

[Demo done on the board to facilitate understanding.]

\end{frame}

\begin{frame}
\frametitle{Other Methods of \texttt{List}}

It may be desirable to add methods for searching, sorting and inserting new items:
\begin{itemize}
	\item \texttt{void Sort( )}
	\item \texttt{ListEntry FindFirst( )}
	\item \texttt{ListEntry FindNext( )}
	\item \texttt{void InsertBefore( )}
	\item \texttt{void InsertAfter( )}
	\item \texttt{bool Contains( )}
\end{itemize}
It should be noted that many of these methods are similar in complexity to the \texttt{DeleteFirst( )} method.

Also, it should be noted that more error checking could be added to the methods that have been presented.

\end{frame}

\begin{frame}
\frametitle{Expanding on the Simple Linked List}

A second reference in each \texttt{ListEntry} that refers to the previous element can simplify deletion and other complex operations.\\
\quad This is known as a doubly-linked list.

The use of dummy items (i.e., dummy head or dummy tail) can simplify some the termination conditions for some loops.

A dummy item is a \texttt{ListEntry} that has a valid reference but never any valid data. It is constructed when the list is constructed.

A length member field can be added to keep track of the length of the list as items are added and removed.

\end{frame}

\end{document}

