\input{configuration}

\title{Lecture 8 --- Loops }

\author{J. Zarnett\\
\texttt{jzarnett@uwaterloo.ca}}
\institute{Department of Electrical and Computer Engineering \\
  University of Waterloo}
\date{\today}

\begin{document}

\begin{frame}
  \titlepage
  
  \begin{center}
  \small{Acknowledgments: W.D. Bishop}
  \end{center}
 \end{frame}
 
\begin{frame}
\frametitle{Loops}

Loops are another kind of control statement: \alert{iteration} statements.

Iteration: the repetition of a group of statements.

Using a loop statement in code means a block of statements is repeated for some number of iterations.

This may be a fixed number or vary based on the state of the program.

\end{frame}

\begin{frame}
\frametitle{Loops: Fixed vs. Variable}

Some examples:

A loop with fixed iterations might repeat a block of statements exactly 10 times.

A loop with a variable number of iterations might repeat a block of statements until the user enters the letter 'Q' to quit.

\end{frame}


\begin{frame}
\frametitle{Loop Terminology}

\alert{Pretest} loops evaluate one or more expressions prior to executing the statement block.

\alert{Posttest} loops execute the statement block once prior to evaluating one or more expressions.

Loops come in a few different varieties:

\begin{itemize}
	\item Counter-controlled loops increment / decrement a counter until the counter reaches a threshold
	\item Logically-controlled loops evaluate an expression and terminate when the expression is false
	\item User-controlled loops can break out of the loop anywhere within the statement block
\end{itemize}

\end{frame}

\begin{frame}
\frametitle{C++ Loop Types}
C++ provides four types of loops:

\begin{itemize}
\item The \texttt{while} loop
\item The \texttt{do-while} loop
\item The \texttt{for} loop
\item The \texttt{foreach} loop (in C++11 and higher)
\end{itemize}

Let's just jump right in and look at the syntax.

\end{frame}

\begin{frame}[fragile]
\frametitle{The \texttt{while} Loop}

This is the syntax for a \texttt{while} loop.

\begin{verbatim}
while ( condition ) 
{
    // Loop Body statements
}
\end{verbatim}

Like the if-statement, \textit{condition} in this is a boolean expression.

The \texttt{while} loop is a pretest loop: the condition is evaluated first.

\end{frame}

\begin{frame}
\frametitle{The \texttt{while} Loop}

Like the if-statement, \textit{condition} is evaluated and if it is true, the block of statements in the \{ \} are executed.

That block of statements is referred to as the \alert{loop body}.

If \textit{condition} is false, the statements of the loop body are not executed.

It may happen that the body of the loop never executes.



\end{frame}

\begin{frame}
\frametitle{The \texttt{while} Loop: Repetition}

What makes the \texttt{while} different from \texttt{if} is the \alert{repetition}.

At the end of the statement body (the \} character), control goes back to the \texttt{while} statement.

The condition is evaluated again.\\
	\quad Same applies: if true, the loop body is executed.
	
This continues until the condition evaluates to false.


\end{frame}


\begin{frame}[fragile]
\frametitle{The \texttt{while} Loop: Countdown}

Here's another example of a \texttt{while} loop:

\begin{verbatim}

int countdown = 10;

while ( countdown > 0 )  {
    cout << countdown << endl;
    countdown--;
}

\end{verbatim}

[Demo: output of this code.]

\end{frame}

\begin{frame}
\frametitle{The Infinite Loop}

What happens if we forget the \texttt{countdown{-}{-};} statement in that loop?

The variable \texttt{countdown} remains at 10 and the \texttt{while} condition will always evaluate to true.

This will go on indefinitely (or until you get frustrated and close the program). The term for this is an \alert{infinite loop}.

\end{frame}

\begin{frame}
\frametitle{The Infinite Loop}

An infinite loop is very often an error condition.

Most programs should terminate at some point.

In some circumstances, however, the infinite loop is intended: the program should never terminate.

Example: the software in your router. On boot up it starts running its program, and continues, never ending (until the plug is pulled).

To create an infinite loop, write \texttt{while( true )}.

\end{frame}

\begin{frame}
\frametitle{The \texttt{break} Statement}

There are two special statements that can be written in the loop body that control the flow of execution.

The first of these is the \texttt{break} statement.

Yes, this is the same keyword as in the \texttt{switch} statement.\\
\quad The context indicates it means something slightly different.

When the break statement executes, it means ``exit the loop now''.

\end{frame}

\begin{frame}
\frametitle{Proper use of \texttt{break}}

A \texttt{break} statement may be used to jump out of a loop, even an infinite one, in the middle of a statement block.

If possible, \texttt{break} statements should be avoided as they can result in code that is more difficult to debug.

The rule of thumb is that the \texttt{break} statement should be used if the code is clearer with it than it would be without it.

Multiple \texttt{break} statements can exist in a loop if multiple conditions for exiting the loop need to be evaluated.

\end{frame}

\begin{frame}[fragile]
\frametitle{While loop with Break}

\begin{verbatim}
int main ( ) {
    int counter = 0;

    while( counter < 500 ) {
    
        counter++;
        if ( counter > 4 ) {
            break;
        }
        cout << counter <<endl;
    }
    return 0;
}
\end{verbatim}

[Demo: output of this program]

\end{frame}

\begin{frame}
\frametitle{Proper Use of Break}
A \texttt{break} statement completely ends the loop, no matter if the loop condition is true or not.

If you write an infinite loop (\texttt{while( true )}) one way to use this properly is to have a condition inside that uses \texttt{break}.

\end{frame}


\begin{frame}[fragile]
\frametitle{While loop with User Input}

\begin{verbatim}
int main ( ) {
   cout << "Enter a negative number to exit." << endl;
   int number = 0;
    
   while (number >= 0) {
       cout << "Enter a number: ";
       int number;
       cin >> number;
   } // End of loop
    
   cout << "Negative number entered." << endl;
   return 0;
}    

\end{verbatim}
\end{frame}

\begin{frame}[fragile]
\frametitle{While loop with User Input \& Break}

Let's rewrite this with \texttt{break}:

\begin{verbatim}
int main ( ) {
   cout << "Enter a negative number to exit." << endl;
   int number = 0;
    
   while (true) {
       cout <<"Enter a number: ";
       int number;
       cin >> number;
       if ( number < 0 ) {
           break;
       }
    }
    cout << "Negative number entered." << endl;
    return 0;
}    

\end{verbatim}
\end{frame}

\begin{frame}
\frametitle{The \texttt{continue} Statement}
The other loop body statement that controls execution is the \texttt{continue} statement.

The \texttt{continue} statement works a lot like the \texttt{break} statement, except instead of exiting the loop, it means ``go back to the start of the loop''.

In the \texttt{while} loop, the condition is tested, and if it's still true, the next iteration of the loop executes.

\end{frame}

\begin{frame}[fragile]
\frametitle{While loop with Continue}

\begin{verbatim}
int main ( ) {
    int counter = 0;

    while( counter < 10 ) {
        counter++;
        if( counter == 4 )
        {
            continue;
        }
        cout << counter << endl;
    }
    return 0;
}
\end{verbatim}

[Demo: output of this program]

\end{frame}

\begin{frame}
\frametitle{Use of Continue}
Like the \texttt{break} statement, \texttt{continue} should be avoided if possible, as they can result in code that is more difficult to debug.

Multiple \texttt{continue} statements can exist in a loop if multiple conditions for going to the next iteration of the loop exist.

If the loop condition is no longer true, use of \texttt{continue} takes us to testing the condition; it will evaluate to false, and the loop ends.

In this way, use of \texttt{continue} may have the same outcome as \texttt{break}, though it gets there by a different path.

\end{frame}

\begin{frame}[fragile]
\frametitle{The \texttt{do-while} Loop}
A variant of the \texttt{while} loop that remains in the language for historical reasons is the \texttt{do-while} loop.

Its syntax is a lot like the \texttt{while} loop:

\begin{verbatim}

do 
{
    // Loop body

} while ( condition );

\end{verbatim}

Note the semicolon that appears after the condition's closing bracket.

\end{frame}

\begin{frame}
\frametitle{The \texttt{do-while} Loop}

Some important things to observe about the \texttt{do-while} loop.

The loop body is preceded by \texttt{do} to indicate the start of the loop.

The condition is checked at the \underline{end} of the loop body, not beginning.\\
\quad This is a posttest loop.

Key observation: the loop body will execute at least once, even if the condition is false.

\end{frame}

\begin{frame}[fragile]
\frametitle{While vs. Do-While}

Let's compare this while loop:

\begin{verbatim}
int count; 
cin >> count;

while ( count > 0 )  {
    cout << count << endl;
    count--;
}
\end{verbatim}

...with this one:

\begin{verbatim}
int count;
cin >> count;
do {
    cout << count << endl;
    count--;
} while ( count > 0 );
\end{verbatim}


\end{frame}

\begin{frame}
\frametitle{The \texttt{do-while} Loop}

It is still possible to write infinite loops with \texttt{do-while}.

Similarly, the \texttt{break} and \texttt{continue} statements work the same way.

The use of \texttt{do-while} is not recommended as it is really only in the language for historical reasons; use the \texttt{while} loop instead.

\end{frame}

\end{document}

